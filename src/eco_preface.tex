%!TEX root = ecosystem.tex

% Generated file. Do not edit.

% Copyright 2024 René Ferdinand Rivera Morell
% Creative Commons Attribution 4.0 International License (CC BY 4.0)

\chapter{Foreword}

ISO (the International Organization for Standardization) is a worldwide
federation of national standards bodies (ISO member bodies). The work of
preparing International Standards is normally carried out through ISO technical
committees. Each member body interested in a subject for which a technical
committee has been established has the right to be represented on that
committee. International organizations, governmental and non-governmental, in
liaison with ISO, also take part in the work. ISO collaborates closely with the
International Electrotechnical Commission (IEC) on all matters of
electrotechnical standardization.

The procedures used to develop this document and those intended for its further
maintenance are described in the ISO/IEC Directives, Part 1. In particular, the
different approval criteria needed for the different types of ISO documents
should be noted. This document was drafted in accordance with the editorial
rules of the ISO/IEC Directives, Part 2 (see \url{https://www.iso.org/directives}).

Attention is drawn to the possibility that some of the elements of this document
may be the subject of patent rights. ISO shall not be held responsible for
identifying any or all such patent rights. Details of any patent rights
identified during the development of the document will be in the Introduction
and/or on the ISO list of patent declarations received
(see \url{https://www.iso.org/patents}).

Any trade name used in this document is information given for the convenience of
users and does not constitute an endorsement.

For an explanation of the voluntary nature of standards, the meaning of ISO
specific terms and expressions related to conformity assessment, as well as
information about ISO’s adherence to the World Trade Organization (WTO)
principles in the Technical Barriers to Trade (TBT)
(see \url{https://www.iso.org/iso/foreword.html}).

This document was prepared by Technical Committee ISO/IEC JTC1, Information
technology, Subcommittee 22, Programming languages, their environments and
system software interfaces, Working Group 21, \Cpp{}.

A list of all parts in the ISO/IEC \disno{} series can be found on the ISO
website.

Any feedback or questions on this document should be directed to the user’s
national standards body. A complete listing of these bodies can be found at
\url{https://www.iso.org/members.html}.
